\documentclass[11pt]{article}
\usepackage[letterpaper, left=.8in, top=0.9in, right=.8in, bottom=0.70in,nohead,includefoot, verbose, ignoremp]{geometry}
%\usepackage{charter} %choose default font ... your choice here % {mathptmx} % mathrsfs} % mathptmx} %mathpple} %mathpazo}
\usepackage{enumerate} % for different labels in numbered lists 
\usepackage{xy}\xyoption{all} \xyoption{poly} \xyoption{knot}  
\usepackage{latexsym,amssymb,amsmath,amsfonts,graphicx,color,fancyvrb,amsthm,enumerate,natbib,movie15}
\usepackage[pdftex,pagebackref=true]{hyperref}
\usepackage[svgnames,dvipsnames,x11names]{xcolor}
\hypersetup{
colorlinks,%
linkcolor=RoyalBlue2,  % colour of links to eqns, tables, sections, etc
urlcolor=Sienna4,   % colour of unboxed URLs
citecolor=RoyalBlue2  % colour of citations linked in text
}
\usepackage{fancybox}
\usepackage{tikz}

%\renewcommand{\includegraphics}{}  % use this to suppress inclusion of figs for proofing

% custom definitions ...
\def\eq#1{equation (\ref{#1})}
\def\pdf{p.d.f.\ } \def\cdf{c.d.f.\ }
\def\pdfs{p.d.f.s} \def\cdfs{c.d.f.s}
\def\mgf{m.g.f.\ } \def\mgfs{m.g.f.s\ }
\def\ci{\perp\!\!\!\perp}                        % conditional independence symbol
\def\beginmat{ \left( \begin{array} }
\def\endmat{ \end{array} \right) }
\def\diag{{\rm diag}}
\def\log{{\rm log}}
\def\tr{{\rm tr}}
%

%% Document starts here ...
%%
\begin{document}
\vspace{-1in}
\title{\bf STA 663 Project - Scalable K-means$++$}
\author{Bai Li, Jialiang Mao}
\maketitle  
 

 
\section*{Abstract}
K-means algorithm is one of the most popular clustering algorithm. A crucial part of k-means algorithm is the choice of initial centers while a poor initial centers may lead to locally optimal solution. To beat this, the k-means++ initialization is proposed to obtain an initial set of centers that is close to the global optimum solution. However, because of its sequential nature, the k-means++ is not scalable. The paper introduces a scalable k-means++ algorithm by reducing the number of passes needed to obtain a good initialization. In this project, we implement this algorithm and test its performance with simulation studies.

  
\section{Introduction}






\section{Implementation}


\section{Testing}

\section{Optimization}

\section{High performance computing}

\section{Application and comparison}
\clearpage
\newpage

\bibliography{reference}
\bibliographystyle{ims}

\section*{Appendix}
\clearpage
\newpage

\end{document}